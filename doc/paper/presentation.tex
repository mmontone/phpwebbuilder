\documentclass[a4paper,10pt]{article}
%\documentclass[a4paper,10pt,draft]{article}

\usepackage{graphicx}
\usepackage[utf8]{inputenc}
\usepackage[spanish]{babel}
\usepackage[left=2cm,top=3cm,right=2cm]{geometry}
\usepackage[nolineno]{lgrind}

\title{\PWB \\ {\normalsize Framework Multipropósito de Aplicaciones Web sobre PHP}}

\date{}
\author{
\begin{tabular}[t]
{c@{\extracolsep{5em}}c}
Alejandro Siri & Mariano Montone \\
\small Eureka Consulting & \small Eureka Consulting \\
\small Calle 11 Nro 684 & \small Calle 11 Nro 684 \\
\small La Plata, 1900, Argentina & \small La Plata, 1900, Argentina \\
\small +54 221 489 5591 & \small +54 221 489 5591 \\
\normalsize asiri@eureka-consulting.com.ar & \normalsize mmontone@eureka-consulting.com.ar
\end{tabular}
}

\newcommand{\comment}[1]{}
\newcommand{\PITS}{\emph{Programming in the Small}} %en.wikipedia.org/wiki/Programming_in_the_small
\newcommand{\PWB}{\emph{PHPWebBuilder}}

\newcommand{\sourcecode}[1]{
\begin{minipage}{12cm}
\begin{lgrind}
\input{#1}
\end{lgrind}
\end{minipage}
}

\begin{document}

\maketitle

\abstract{
%El enfoque puede ser puramente tecnico, comentando los como, por que, quienes, cuanto y cuando hicieron el proyecto. Desde la elección de herramientas, problemas tecnicos encontrados, etc. o lo que consideren.
%
%También, si lo tiene, se puede incluir algun condimento respecto de que utilidad/visión (social/empresarial/politica) aporta el proyecto.
El diseño y desarrollo de aplicaciones (tanto de escritorio como web) plantea muchos problemas reiterativos. Existen múltiples soluciones para cada una de ellas, cargando sobre el programador la responsabilidad de seleccionarlas e integrarlas. \PWB\ es un Framework Open Source Orientado a Objetos escrito en PHP para el desarrollo de aplicaciones Web, tanto de Internet como de Intranet, que integra soluciones a cada uno de estos problemas, elegidas en base a la experiencia de la empresa en el desarrollo de este tipo de aplicaciones.

{\bf Keywords}: Templates declarativos, desarrollo por componentes, mapeo objeto-relacional, persistencia por alcance, AJAX, Comet, compilación, macros.

}

% \section{Nombre}
% En un principio, Perseus fue llamado PHPWebBuilder (porque era utilizado para hacer webs en PHP). Luego de muchas modificaciones al framework, decidimos
% rebautilzarlo. El nombre Perseus fue elegido por la mitología griega, ya que este era un heroe que, con ayuda de muchas herramientas, pudo acabar
% al monstruo Medusa, que convertía a sus contrincantes en piedra.
%
% En un principio, Perseus fue llamado PHPWebBuilder (porque era utilizado para hacer webs en PHP). Luego de muchas modificaciones al framework decidimos rebautilzarlo. El nombre Perseus fue elegido por la mitología griega, ya que este era un heroe que, con ayuda de muchas herramientas, pudo acabar al monstruo Medusa, que convertía a sus contrincantes en piedra.
%
% Perseus, gracias a la integración de sus múltiples herramientas, sirve para atacar desde proyectos pequeños hasta los más grandes, que petrifican hasta a los mejores programadores.
%
% \section{Origenes}
%
% En un principio, Perseus fue diseñado para desarrollo de sitios web. Las ventajas principales las presentaba en cuanto a lo que es persistencia y mapeo Objeto-Relacional bajo PHP, además de contar con un CMS con permisos de acceso.
%
% Luego, se le implementó el framework de componentes, templates y de eventos, con lo que se pudo empezar con los primeros sistemas de intranet.
%
% En las últimas versiones, con el manejo de macros, oql, y compilación, los proyectos se simplifican cada vez más.
%
% \section{Tecnologías de soporte}
%
% La elección inicial de LAMP, para aplicaciones web, era evidente, por el amplio alcance de estas tecnologías en los servidores comerciales web.
%
% Ante la mayor cantidad de opciones para hacer software de escritorio e intranet, nos vimos inclinados por las tecnologías web, por la facilidad de distribución y la escalabilidad de la misma.
%
% Al haber elegido LAMP nos vimos forzados a desarrollar en PHP. Si bien tiene limitaciones, PHP resultó tener buen soporte, además de ser un lenguaje flexible, lo que permite hacer diseños fácilmente modificables.
%
% }

\section{Introducción}

El diseño y desarrollo de aplicaciones (tanto de escritorio como web) plantea muchos problemas reiterativos. La interacción con el usuario, la presentación de la información, el almacenamiento y la recuperación de la misma son algunos de los problemas más frecuentes y que más tiempo consumen.
Por esto mismo han recibido mucha atención, habiéndose conseguido a lo largo de los años múltiples soluciones. Seleccionar e integrar éstas queda a cargo del programador o grupo de trabajo de un proyecto específico.

\PWB \ es un framework de desarrollo que integra múltiples soluciones que, basados en la experiencia de la empresa, mejoran el tiempo de desarrollo y la calidad del producto de software sin sacrificar flexibilidad en el diseño.

%\section{\PWB}

El framework está diseñado bajo una arquitectura MVC (Model-View-Controller)\cite{mvc}.
Esto quiere decir que una aplicación se compone de 3 capas:
\begin{itemize}
\item El \emph{Modelo}. Es la representación de la información de dominio específico de la aplicación.
\item El \emph{Controlador}. Está basada en la programación de componentes. Responden a la acción del usuario y definen los aspectos navegacionales de la aplicación (más sobre ésto en la sección \ref{sec-controller}).
\item La \emph{Vista}. La forma en la que se presentan los datos y botones que el usuario ve, se define a través de templates declarativos, HTML y CSS. Desarrollamos los aspectos de presentación en la sección \ref{sec-view}.
\end{itemize}

Existe además otra parte que integra al framework. Esto es la ``Programación en lo pequeño'' (\PITS), la programación de los módulos y funciones del programa, que sirven para unir estas capas. Los frameworks actuales ayudan a simplificar el desarrollo bajo MVC; el soporte para \PITS \ generalmente es dependiente del lenguaje/plataforma. \PWB \ se apoya en PHP para parte de esta tarea, y además provee al desarrollador herramientas para solucionar o simplificar las tareas en las otras áreas:

\begin{itemize}
\item Para el modelo, presenta un mapeo automático de base de datos (sección \ref{sub-pers}), un lenguaje de consultas complejas (sección \ref{sub-oql}) que tiene en cuenta la herencia del modelo de clases, y autogeneración del esquema de base de datos (sección \ref{sub-adapt}). Esto permite que la tarea del desarrollador se limite a enfocarse a resolver la problemática que plantea el diseño del modelo.

\item Para el controlador, utiliza un sistema de componentes (sección \ref{sub-comp}), que permiten reutilizar ``partes de aplicación'' , ya sea en diferentes partes de una misma aplicación o en diferentes proyectos. Además, los widgets (sección \ref{sub-widget}) simplifican y encapsulan la interacción con el usuario.

\item Para la vista, el sistema de templates (sección \ref{sub-templates}) basados en XML permiten una adaptación directa del trabajo de un diseñador gráfico (sección \ref{sub-templates-adapt}). El sistema se encarga de presentar la información utilizando AJAX de manera transparente, u otra forma de rendereo (sección \ref{sub-render}) (esto es fácilmente configurable).

\item Por último, para \PITS, \PWB \ presenta características disponibles en otros lenguajes y plataformas pero no existentes en PHP. Estas son la implementación de Eventos (sección \ref{sub-events}),
%Weak References (sección \ref{sub-weak}),
%creación de DSLs (sección \ref{sub-phpcc}),
%Mixins (sección \ref{sub-mixins}).
y Macros (sección \ref{sub-macros}).

\end{itemize}

\section{Características del Modelo}

\subsection{Persistencia}
\label{sub-pers}
La persistencia del modelo se hace mediante la subclasificación de una clase especial (PersistentObject).
Esta clase provee los métodos necesarios para describir la metadata en su inicialización, y también para las
operaciones de guardado y borrado.
Luego, para la recuperacion de objetos, existe la clase Report, que permite obtener los objetos de una
coleccion, filtrada por varios criterios y ordenada. Estos reportes tienen en cuenta las subclases de los
objetos, y utilizan herencia para las variables de instancia.

\subsection{OQL}
\label{sub-oql}
Dada la complejidad de construir un reporte completo a mano, desarrollamos un lenguaje OQL para consulta de
los objetos, que construye un reporte.

El lenguaje está íntimamente relacionado con el código PHP, ya que puede utilizar las variables en el scope.
Además, puede hacer consultas utilizando la identidad de los objetos.
\subsection{Adaptación de base de datos}
\label{sub-adapt}
En el desarrollo del modelo, incluímos todo lo que es persistencia. Establecemos el los tipos de las variables
de cada clase, y las relaciones de subclasificación.

De este modelo, la información del esquema de persistencia para una base de datos relacional se puede deducir.
Por esto, también podemos generar el esquema automáticamente, liberando de esta carga al programador.

Además, podemos verificar que el esquema de base de datos sea el correcto, y mostrar y ejecutar las correcciones
necesarias. Esto también es muy útil cuando se hace una modificacion en el modelo de una aplicación, y se necesita
adaptar la base de datos a los nuevos cambios.
\section{Características del Controller}

\subsection{Componentes}
\label{sub-comp}

Las aplicaciones se construyen mediante ensamblado de componentes. Estos componentes son reutilizables,
ya que pueden ser parametrizados, pueden levantar eventos, y pueden contener otros componentes.

Un componente puede llamar a otro para realizar una tarea (por ejemplo, un componente Welcome, ante un click
del usuario en un botón "Ingresar", puede llamar al componente Login), y esperar la respuesta de este
componente para continuar su ejecución (cuando el componente devuelve el usuario validado).
\subsection{Widgets}
\label{sub-widget}

Existen unos componentes especiales, llamados Widgets, que permiten interactuar con el usuario (por ejemplo
con el componente Input, que permite recibir un string del usuario, el Text, que le presenta un texto, o el
CheckBox, que presenta un checkbox).

\subsection{Multiple Dispatching y Context Dispatching}
\label{sub-dispatch}
Otra característica llamativa, es el múltiple dispatching de funciones.
Dado que muchas veces el método a utilizar depende de más de una clase, las funciones de múltiple dispatching nos ayudan a resolver este problema.

Primero se define la función, con los parametros a utilizar tipados, y luego se hace el llamado, que utiliza los tipos de los parámetros para
resolver el método a utilizar

Además, las funciones de múltiple dispatching pueden ser utilizadas pasando el contexto (la rama de componentes dentro de la que se hace el llamado)
de aplicación, para de esta manera poder responder de manera diferente a un evento, dependiendo del contexto.

Por ejemplo, podemos tener un componente TaskList que muestre las tareas realizada en un día. Para cada tarea muestra la descripción, la hora, y el
nombre de quién la completo en un componente TaskList. Si luego quisieramos ver las tareas realizadas por uno de los usuarios, nos interesaría ver
la hora, la descripción, pero no el nombre del usuario, porque ya se sabe por el contexto. Podríamos usar en ese caso un componente UserTaskShow.

Para implementar esa diferencia, una forma común sería tener 2 componentes, TaskList y UserTaskList. Esto duplica código, y además es difícil de
implementar cuando el contexto de los componentes es de varios componentes (ya que tendríamos que hacer un nuevo componente para cada elemento en
la rama de componentes hasta llegar al que realmente implementa la diferencia).

Entonces, podemos utilizar el Context Dispatching para elegir un componente UserTaskShow, en lugar del otro, el TaskShow, independientemente del nivel de
anidamiento del componente que hace el dispatch y el componente que provee el contexto.


\section{Características de la Vista}

\subsection{Templates}
\label{sub-templates}
Para la generación de interfaz, utilizamos la técnica conocida de los templates.
Por comodidad, agregamos las siguientes características:

- Los templates se "heredan", así que un Componente subclase de otro que tiene template, hereda de este el
template (siempre y cuando no tenga uno más específico). También son heredables por Mixins.

- Los templates son declarativos: No incluyen comandos ni iteradores (como tienen engines como el Smarty).
De esta manera, el control de la aplicación está 100\% en los componentes, y además nos aseguramos que los
templates generen un XML bien formado.

- Los templates están basados en XML, con un par de tags extras (<template> y <container>), así que se puede
generar cualquier XML (como XHTML) para mostrar los componentes.

Además, cada componente tiene un template default, por lo que no se necesita crearle uno para tener la aplicación
funcionando.

\subsubsection{Adaptación de Diseños existentes}
\label{sub-templates-adapt}
Debido a que los templates son XML, se puede tomar una página en HTML y agregarle los tags <template> y <container> donde
se quiera, y de esta manera conseguir un template a muy bajo costo.
Además, como los templates se heredan, utilizando clases bien acomodadas, una aplicación queda con un diseño completo en
solo 3 o 4 templates.


\subsection{Renderings}
\label{sub-render}
El rendereo de la aplicación se hace 100\% mediante PWB, y además todo lo generado es XML bien formado, de
modo que de manera transparente podemos renderear en AJAX.

La manera de cambiar el engine de rendereo es tan simple como cambiar en el archivo de configuración, donde
dice page\_renderer=StandardPageRenderer por page\_renderer=AjaxPageRenderer, o incluso por page\_renderer=CometPageRenderer
o page\_renderer=XULPageRenderer

\subsubsection{Comet}

Comet es una tecnología similar a AJAX, que se caracteriza por mantener una conexión abierta con el servidor
en todo momento. PWB aprovecha esto, ya que un importante tiempo de procesamiento de los scripts PHP es la
carga del script y los datos de la sesión de usuario, que en este caso quedan vivos en la memoria del servidor.

Mediante pasaje de mensajes de otro script, que envía los datos a la aplicación PWB ejecutándose, podemos
mantener la sesión en memoria, mejorando los tiempos de respuesta, e incluso modificando la aplicación del
usuario no solamente cuando éste genera un evento (un click del mouse), sino también cuando, por ejemplo, la base
de datos es modificada.

\subsubsection{XUL}

El proyecto Mozilla incluye un subproyecto llamado XUL. XUL es un lenguaje de definición de interfaces desktop
en XML. Dado que la salida de la aplicación debe ser un XML bien formado, y que la interacción usuario-interfaz
se hace mediante javascript, PWB soporta un XUL Page Renderer, que renderea aplicaciones en XUL. La
diferencia para el usuario, es en los tags de los templates, ya que debe utilizar elementos XUL en lugar de
HTML. Dado que los templates de HTML llevan extensión .xml, y los de xul .xul, la misma aplicación, con tener
templates de los 2 tipos para cada componente, puede ser rendereada de las 2 maneras.

\section{Características del PITS}

\subsection{BugNotifier}

PWB incluye un BugNotifier, que automatiza el manejo de errores "no manejados", permitiendo al usuario que
encuentra una condición de error dentro de la aplicación, enviar el reporte de error a un mail configurado a
tal fin, y también reiniciar la aplicación para continuar utilizándola.
%PITS

\subsection{Eventos}
\label{sub-events}
PWB implementa un mecanismo simple de eventos. Todo objeto PWBObject implementa los mensajes addInterestIn, que permite a otro objeto escuchar un evento,
y triggerEvent, que acciona un evento.

Los widgets además accionan eventos en cada acción del usuario, y los objetos del modelo cuando son modificados.


\subsection{Weak References}
\label{sub-weak}
Cuando un objeto \$x queda escuchando un evento de un PWBObject \$y, este último necesita guardar una referencia al primero. En caso de que, por el flujo de la
aplicación, \$x deje de ser necesario, el mecanismo de garbage collection de PHP no puede descartarlo, porque \$y lo conserva referenciado, aunque no lo
necesite realmente.

Por esto implementamos un mecanismo de Weak References, en donde \$y se queda con una referencia de \$x, pero de la cual el garbage collector no se entera,
permitiendo borrar a \$x en caso de que sea necesario.


\subsection{PHPCC}
\label{sub-phpcc}
Para crear el lenguaje OQL, debimos crear un Compiler Compiler para PHP (otros conocidos son Bison, yacc).
De esta manera ahora también se puede extender el framework con múltiples DSLs.

\subsection{Macros}
\label{sub-macros}
Otra funcionalidad interesante, es la utilización de macros. Como se ve en lenguajes como C, las macros pueden \
llegar a tener un rol muy importante en un proyecto, ya que son otra forma de modularización.

Dentro de PWB mismo, tenemos varios tipos de macros:

Las macros, lam, y select, que permiten hacer un pre-procesamiento del código, generando código PHP con
mayor funcionalidad:
\verb"#@ select u:User where u.name='pepito'@#"
genera un objeto reporte completo, con todos los usuarios llamados pepito, en código PHP.

Y por otro lado, las macros para validación de código, que hacen tests, útiles para el momento del desarrollo
pero consumidoras de tiempo valioso en tiempo de deployment.

Estas son typecheck, y check:
\verb"#@typecheck $c: Component, $u: User@#"
que chequea que las variables \verb"$c" y \verb"$u" sean un Component y un User, respectivamente. En las opciones de
configuración de la aplicación, habilitamos o dehabilitamos el typechecking, y el chequeo no se incluye.

\subsection{Mixins}
\label{sub-mixins}
Otra funcionalidad interesante, es la de los mixins. Muchos lenguages la ya incluyen, y nosotros se la
implementamos (utilizando macros).

Un mixin es una forma de agrupar funcionalidad y agregársela a múltiples clases de objetos, sin que estas
clases estén conectadas por subclasificación. Es lo que vemos como el mejor trade-off entre simple y múltiple
herencia.
\begin{verbatim}
#@mixin ValueHolder {
  var $value;
  function getValue(){
    return $this->value;
  }
  function setValue($value){
    $this->value = $value;
  }
}@#

class Contador{
  #@use_mixin ValueHolder@#
  function increment(){
    $this->setValue($this->getValue()+1);
  }
}
\end{verbatim}
\subsection{Múltiples Configuraciones}

Para el desarrollo colaborativo, es útil mantener múltiples configuraciones. O para trabajar en distintos
clientes. O para hacer testing y deployment.

Por todo esto, PWB mantiene un archivo de configuraciones múltiples, para adaptar cada una a una necesidad
específica.

\subsection{Compilación}

El uso de macros, y la inclusión de los muchos archivos de PWB, hacen que la carga en cada request pueda ser
muy lenta. Por eso desarrollamos un método de compilar los archivos PHP (para que las macros ya estén procesadas),
y además, habilitamos varias formas de compilación del código (todo a un sólo archivo PHP, sólo las clases
utilizadas a un archivo PHP, en archivos separados). Esta configuración es seteable desde los archivos de
configuración.



%\input{caracteristicas.lgrind.tex}

%\section{Trabajos Hechos}
%Entre los trabajos que hicimos, se encuentran sitios web y sistemas de intranet.

\section{Related Work}

Gran parte de nuestra inspiración del ``Controller'' vino de SeaSide \cite{seaside}, un Framework de AppsWeb en Smalltalk.

Las ideas de los Componentes, eventos, weakreferences y callbacks entre Componentes, fueron tomadas de Seaside.

Dejamos de lado el Smalltalk, usando PHP, y el ``HTML Programático'', en lugar de los cuales utlizamos nuestros ``Templates Declarativos'' .

Smalltalk tiene Traits \cite{traits}, de los que nosotros inspiramos nuestros Mixins. En realidad, los mixins son conceptos anteriores, similares a nuestros mixins. Los Traits no tanto.
\begin{verbatim}
Widgets http://en.wikipedia.org/wiki/GUI_Widget

El proyexto XUL www.mozilla.org/projects/xul/. Los renderers transparentes.

Yacc dinosaur.compilertools.net/yacc/index.html, Bison www.gnu.org/software/bison/ y Parsec www.cs.uu.nl/~daan/parsec.html.
OQL http://www.odmg.org/. DSLs http://homepages.cwi.nl/~arie/papers/dslbib/.

Persistencia: Hibernate www.hibernate.org/, Por alcance: JDO www.jpox.org/.

Macros. http://en.wikipedia.org/wiki/C_preprocessor

Compilación. http://www.phpclasses.org/browse/package/3215.html

Zope, http://www.symfony-project.com/,

http://www.phpmvc.net/,

http://www.cakephp.org/,

http://seagull.phpkitchen.com/,

http://www.struts4php.org/,

http://www.mojavelinux.com/projects/studs/.

http://dev.helma.org/
\end{verbatim}



\section{Conclusiones}

\PWB\ no presenta en sí conceptos novedosos, sino que reúne las opciones que consideramos mejores, y los integra e interrelaciona dentro del mismo framework. Esto libera a los desarrolladores de muchas decisiones reiterativas, y les permite enfocarse en los problemas específicos de la aplicación a desarrollar.

Los próximos pasos a realizar son: implementar mejoras en el manejo de colecciones en el modelo, e implementar persistencia por alcance \cite{jpox,jdo}.

%Aún quedan cosas por hacer. Algunas de ellas son implementar mejoras en el manejo de colecciones en el modelo, documentar, agregar soporte DSLs composicionales, mejorar la integración y comunicación de múltiples aplicaciones y diseñar e implementar templates paramétricos y dependientes del contexto y implementar persistencia por alcance similar a JDO \cite{jpox}.


%\section{Agradecimientos}
\bibliographystyle{unsrt}
\bibliography{references}

\comment{
\setcounter{section}{0}
\newpage
\section*{Apéndices}
\section{Código completo}

Modelo:

\sourcecode{src/Post.class.php.tex}
\sourcecode{src/Tag.class.php.tex}

Controller:

\sourcecode{src/PostItem.class.php.tex}
\sourcecode{src/PostList.class.php.tex}
\sourcecode{src/BlogComponent.class.php.tex}
}
\end{document}