\documentclass[a4paper,10pt]{book}

\usepackage{graphicx}
\usepackage[utf8]{inputenc}
\usepackage[spanish]{babel}
\usepackage{html,makeidx}

\title{PHPWebBuilder}
\date{}
\author{Eureka Consulting - \htmladdnormallink{www.eureka-consulting.com.ar}{http://www.eureka-consulting.com.ar} \\
		Alejandro Siri asiri@eureka-consulting.com.ar\\
		Mariano Montone mmontone@eureka-consulting.com.ar}
%\author{}

\begin{document}

\maketitle

\oddsidemargin -.60cm
\evensidemargin -.60cm
\headheight 50pt
\topmargin -1.0cm
\textheight 22cm

\chapter{Abstract}
%El enfoque puede ser puramente tecnico, comentando los como, por que,
%quienes, cuanto y cuando hicieron el proyecto. Desde la elección de
%herramientas, problemas tecnicos encontrados, etc. o lo que consideren.
%
%También, si lo tiene, se puede incluir algun condimento respecto de que
%utilidad/visión (social/empresarial/politica) aporta el proyecto.

PWB sirve para programar, por ejemplo, Vista 2.

\chapter{Tecnologías de soporte}: LAMP->PHP, Mysql
\chapter{Origenes}: Web->Applications
\chapter{Características (con ejemplo)}
\chapter{Componentes}
\section{Persistencia}
\section{Macros}
\section{Compilación}
\section{Renderings}



\chapter{Ejemplo Completo}
\chapter{Trabajos Hechos} En abstracto.
\chapter{Comparaciones}
Zope,
http://www.symfony-project.com/,
http://www.phpmvc.net/,
http://www.cakephp.org/,
http://seagull.phpkitchen.com/,
http://www.struts4php.org/,
http://www.mojavelinux.com/projects/studs/.

\end{document}